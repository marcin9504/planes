\documentclass{article}
\usepackage{polski}
\usepackage[utf8]{inputenc}
\usepackage[OT4]{fontenc}
\usepackage{graphicx,color}
\usepackage{url}
\usepackage[pdftex,hyperfootnotes=false,pdfborder={0 0 0}]{hyperref}
\usepackage{float}

\begin{document}
\thispagestyle{empty} %bez numeru strony

\begin{center}
{\large{Sprawozdanie z laboratorium:\\
Komunikacja człowiek–komputer}}

\vspace{3ex}

Samolot

\vspace{3ex}
{\footnotesize\today}

\end{center}


\vspace{10ex}

Prowadzący: dr hab.~inż. Maciej Komosiński

\vspace{5ex}

Autorzy:
\begin{tabular}{lllr}
\textbf{Marcin Drzewiecki} & inf122472 & marcin.drzewiecki@student.put.poznan.pl \\
\textbf{Adam Pioterek} & inf122446 & adam.pioterek@student.put.poznan.pl \\
\end{tabular}

\vspace{5ex}

Zajęcia czwartkowe, 15:10.

\vspace{35ex}

\noindent Oświadczam/y, że niniejsze sprawozdanie zostało przygotowane wyłącznie przez powyższych autora/ów,
a wszystkie elementy pochodzące z innych źródeł zostały odpowiednio zaznaczone i~są cytowane w bibliografii.  

\newpage



\section*{Udział autorów}
\begin{itemize}
\item Marcin Drzewiecki zaimportował obrazy, napisał podstawowy algorytm do wykrywania obiektów pierwszoplanowych oraz stworzył mozaikę przetworzonych obrazów.
\item Adam Pioterek udoskonalił podstawowy algorytm wykrywania obiektów pierwszoplanowych, napisał algorytm oznaczający kolorowe kontury na zdjęciu oraz znajdujący centroidy znalezionych obiektów.
\end{itemize}

\section{Opis ćwiczenia}
Pierwsza część naszego ćwiczenia polegała na zaimportowaniu zdjęć samolotów na tle nieba.
Na ich podstawie należało wyodrębnić samoloty od tła oraz utworzyć mozaikę obrazów zawierającą kontury samolotów. 
Obrazy powinny zawierać jedynie czarne piksele należące do konturów samolotów, natomiast pozostałe powinny pozostać białe.
Największą trudność w tej części ćwiczenia sprawiło nam znalezienie odpowiedniej kombinacji filtrów do wyodrębnienia konturów.

Druga część polegała na nałożeniu kolorowych konturów samolotów na oryginalne zdjęcia.
Każdy kontur miał mieć różny kolor.
Dodatkowo, na każdym z samolotów należało zaznaczyć centroid.
Najtrudniejszym elementem tej części było przypisanie znalezionych konturów odpowiednim samolotom.

Do wykonania zadania wykorzystane zostały następujące filtry:
\begin{description}
\item[sobel_h oraz sobel_v] znajdują odpowiednio poziome i pionowe krawędzie na zdjęciu;
na wejściu otrzymują zdjęcie w odcieniach szarości (macierz 2D), na wyjściu dają macierz 2D zawierającą mapę krawędzi.
\item[gaussian] przeprowadza rozmycie Gaussa;
na wejściu otrzymuje zdjęcie w odcieniach szarości (macierz 2D) i odchylenie standardowe, na wyjściu daje macierz 2D – przefiltrowane zdjęcie.
\item[dylatacja] – morfologiczna dylatacja, która powiększa jasne rejony na rzecz ciemnych;
na wejściu otrzymuje zdjęcie w odcieniach szarości (macierz 2D), na wyjściu daje macierz 2D.
\end{description}
oraz funkcje:
\begin{description}
\item[label] – separuje cechy na zdjęciu;
na wejściu otrzymuje macierz 2D gdzie jako 0 oznaczone jest tło, na wyjściu daje macierz 2D gdzie każda cecha jest oznaczona inną liczbą.
\item[find_boundaries] – znajduje granice pomiędzy cechami;
na wejściu otrzymuje macierz 2D z odseparowanymi cechami (każda oznaczona inną liczbą), na wyjściu zwraca macierz 2D z oznaczonymi granicami.
\end{description}
 
\section{Wynik działania programu}

\subsection{Kontury}
\begin{figure}[H]
\begin{center}
\includegraphics[width=1\textwidth]{../basic.pdf}
\end{center}
\caption{Mozaika z konturami samolotów}
\label{fig: wykres1}
\end{figure}

\subsection{Kolorowe kontury i centroidy}
\begin{figure}[H]
\begin{center}
\includegraphics[width=1\textwidth]{../advanced.pdf}
\end{center}
\caption{Mozaika z kolorowymi konturami samolotów oraz centroidami}
\label{fig: wykres2}
\end{figure}

\end{document}
